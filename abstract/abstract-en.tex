%!TEX root = ../dissertation.tex

\begin{otherlanguage}{english}
\begin{abstract}
% Set the page style to show the page number
\thispagestyle{plain}
\abstractEnglishPageNumber
% What is SDN & OF
The architectural principles of \gls{SDN} and its most prominent supporting protocol - \emph{OpenFlow} - keep gaining momentum.
\gls{SDN} relies essentially on the decoupling of the \emph{control plane} from the \emph{data plane}, placing the former in a logically centralized component to be executed on commodity hardware - the \emph{\gls{SDN} controller}.
% How OF works & associated limitations
OpenFlow's reactive programming enables the programming of the network based on real-time decisions taken as new traffic hits the data plane, but it requires the first packet of every new flow traversing any \gls{SDN} controlled device to be sent to the controller and evaluated, which when considered in large network environments becomes too large to be handled by a single \gls{SDN} controller instance.\\
% My work
This document describes a new architectural proposal for an elastic \gls{SDN} controller cluster as a solution to overcome the aforementioned limitations, along with proof of concept implementation of said architecture by extending the Floodlight controller and integrating it with the Linux Virtual Server project.
% Keywords
\begin{flushleft}

\keywords{Software-Defined Networking, OpenFlow, Network Function Virtualization, Network Management, Distributed \gls{SDN} Controller}

\end{flushleft}

\end{abstract}
\end{otherlanguage}
