%!TEX root = ../dissertation.tex

\begin{otherlanguage}{portuguese}
\begin{abstract}
\abstractPortuguesePageNumber

% What is SDN & OF
O padrão arquitetural Software-Defined Network \gls{SDN} e o seu protocolo mais proeminente - \emph{OpenFlow} - continuam a ganhar ímpeto.
A arquitetura \gls{SDN} tem por base o desacoplamento do \emph{control plane} do \emph{data plane}, colocando o primeiro num novo componente logicamente centralizado a ser executado em hardware de comodidade - o \emph{Controlador \gls{SDN}}.
% How OF works & associated limitations
O modo de programação reativa do OpenFlow permite a programação da rede em tempo real, tomando decisões de encaminhamento conforme o tráfego dá entrada no \emph{data plane}, sendo necessário para tal que a primeira trama de cada fluxo que atravesse um qualquer dispositivo de rede gerido por um controlador \gls{SDN} seja reencaminhado para o controlador para que seja inspecionado.
Embora o modo reativo proporcione um método mais conveniente e flexível para programar a rede quando comparado com o modo proactivo, o custo computacional associado à execução das tarefas necessárias torna-se incomportável para ser executado por uma única instância do controlador \gls{SDN} quando aplicado a redes de grande dimensão.\\
% My work
Propõe-se neste documento uma nova arquitetura que visa tornar o controlador \gls{SDN} num cluster elástico como solução para o problema de escalabilidade apresentado, tendo sido esta mesma arquitetura implementada através da exntensão do controlador Floodlight e da integração com o projecto Linux Virtual Server.
% Keywords
\begin{flushleft}

\keywords{Software-Defined Networking, OpenFlow, Network Function Virtualization, Gestão de redes, \gls{SDN} Controller Distribuído}

\end{flushleft}

\end{abstract}
\end{otherlanguage}
