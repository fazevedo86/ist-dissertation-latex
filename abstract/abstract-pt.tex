%!TEX root = ../dissertation.tex

\begin{otherlanguage}{portuguese}
\begin{abstract}
\abstractPortuguesePageNumber
% What is SDN & OF
O padrão arquitetural Software-Defined Network \gls{SDN} e o seu protocolo mais proeminente - \emph{OpenFlow} - continuam a ganhar ímpeto.
A arquitetura \gls{SDN} tem por base o desacoplamento do \emph{control plane} do \emph{data plane}, colocando o primeiro num novo componente logicamente centralizado a ser executado em hardware de comodidade - o \emph{Controlador \gls{SDN}}.
% How OF works & associated limitations
O OpenFlow suporta programação em modo proactivo e reativo.
O modo proactivo é baseado na pré-programação estática do \emph{data plane}.
O modo reativo permite a programação da rede em tempo real, tomando decisões de encaminhamento conforme o tráfego dá entrada no \emph{data plane}, sendo necessário para tal que a primeira trama de cada fluxo que atravesse um qualquer dispositivo de rede gerido por um controlador \gls{SDN} seja reencaminhado para o controlador para que seja inspecionado.
A ação apropriada para o fluxo é então programada no dispositivo de rede e a trama é reencaminhada de novo para o \emph{data plane}.
Embora o modo reativo proporcione um método mais conveniente e flexível para programar a rede quando comparado com o modo proactivo, o custo computacional associado à execução das tarefas necessárias torna-se incomportável para ser executado por uma única instância do controlador \gls{SDN} quando aplicado a redes de grande dimensão.
% How OF works & associated limitations
Uma forma de contornar esta limitação passa pela existência de várias instâncias de controlador \gls{SDN}, sendo atríbuido a cada instância um subconjunto dos dispositivos de rede.
Contudo, esta solução requer uma configuração complexa, trabalhosa e volátil tanto no controlador \gls{SDN} como nos dispositivos de rede por forma a manter a política de rede consistente e conseguir-se distruição de carga eficiente entre as instâncias de controlador \gls{SDN} existentes.
% My solution
Para que seja possível tirar o máximo partido do potencial do OpenFlow propõe-se então uma nova arquitetura para o controlador \gls{SDN} baseado em OpenFlow, centrada no conceito de \emph{cluster} elástico e no paradigma de comunicação \emph{anycast}.
% Keywords
\begin{flushleft}

\keywords{Software-Defined Networking, OpenFlow, Network Function Virtualization, Gestão de redes, \gls{SDN} Controller Distribuído}

\end{flushleft}

\end{abstract}
\end{otherlanguage}
