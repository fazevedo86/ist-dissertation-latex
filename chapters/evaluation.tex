%!TEX root = ../dissertation.tex

\chapter{Evaluation}
\label{chapter:evaluation}
Ultimately the goal to be achieved by this work is an architecture for a network infrastructure that while solving existing manageability issues improves the overall performance of OpenFlow-based \gls{SDN} controllers operating in reactive mode.
This chapter reproduces the results obtained from various tests to which the proof of concept implementation described in chapter \ref{chapter:implementation} was subjected to in order to validate that the proposed architecture described in chapter \ref{chapter:architecture} met its objects.\\
The scope of the testing ranges from strictly functional tests, in which the correctness of architectural model and of the implementation are validated, to performance tests using both the proof of concept implementation and the base Floodlight implementation.
%
\section{Test environment}
\label{section:test-environment}
A virtual test environment was instantiated for the evaluation of this work resorting to a shared \gls{IaaS} provider.
This environment was composed of six \glspl{VM}, which where allocated for the following purposes:
\begin{itemize}
	\item \textbf{1 \gls{VM}} was allocated to execute the Mininet network emulator
	\item \textbf{2 \glspl{VM}} were allocated to the execute the Request Router component described in chapter \ref*{chapter:state-of-the-art} section \ref{section:request-router} and chapter \ref*{chapter:implementation} section \ref{section:request-router-implementation}
	\item \textbf{3 \glspl{VM}} were allocated to execute \gls{SDN} controller instances described in chapter \ref*{chapter:state-of-the-art} section \ref{section:SDN-controller-cluster} and chapter \ref*{chapter:implementation} section \ref{section:SDN-controller-cluster-implementation} as well as instances of version 1.1 of the Floodlight controller.
\end{itemize}
%
The Operating System chosen to run in these \glspl{VM} was the version 8 of the Debian Linux distribution seeing that it provided a lightweight environment built on top of stable versions of the libraries required to execute the components supporting the implementation.\\
Due to the lack of resources to emulate a complete management network infrastructure, it was not possible to test the routing integration described in chapter \ref*{chapter:implementation} section \ref{subsection:anycast-implementation}.
However, seeing that this implementation uses only proven components and protocols - Quagga, \gls{BGP} and \gls{BFD} - it is not expected that any problem arises from this part of the request router component.
%
\subsection{Network testbed}
\label{subsection:Mininet}
Because the test environment was restricted to a virtualized support, it was also necessary to provide a virtualized network testbed solution.\\
Mininet is a network emulator written in python, that is capable of emulating switches and hosts in custom defined topologies, by employing process-based virtualization and making use of Linux's network namespaces \cite{mininet}.
Because its emulated switches are able to support the OpenFlow protocol, Mininet is the network emulator of choice for \gls{SDN} testbeds.\\
Version 2.2.1 of Mininet was used to emulate the different network topologies used in the tests described in this chapter.
%
*TODO: Add diagrams of the network topologies used
%
%
\section{Functional tests}
\label{section:functional-tests}
The first batch of tests carried out were functional tests, designed to validate the correctness of the implementation and the validity of the architecture.\\
%
For the validation of the \gls{SDN} controller elastic cluster described in chapter \ref*{chapter:implementation} section \ref{section:SDN-controller-cluster-implementation}, the correctness of such implementation greatly depended on the correctness of the Floodlight controller, and it is therefore considered that any behavior coherent with that of the base version of Floodlight is therefore correct.
Furthermore, all architecture-specific behavior must be validated with the requirements and desired behavior described in chapter \ref*{chapter:architecture} \ref{section:SDN-controller-cluster}.\\
%
The request router, while partially implemented as stated in section \ref{section:test-environment}, was also subject to testing to the remaining components that were implemented in the test environment.
The validation of the correctness of the request router is validated with the requirements and desired behavior described in chapter \ref*{chapter:architecture} section \ref{section:request-router} and implementation specific particularities indicated in chapter \ref*{chapter:implementation} section \ref{section:request-router-implementation}.
%
%
\subsection{Scenarios}
\label{subsection:functional-tests-scenarios}
For the \gls{SDN} controller elastic cluster specific functional tests, the key points tested where as follows:
\begin{enumerate}
	\item Cluster membership behavior and exchanged messages
	\item Cluster membership consistency throughout the cluster upon membership changes
	\item \gls{MIB} consistency throughout the cluster and exchanged messages
	\item \gls{MIB} consistency throughout the cluster upon cluster membership changes
	\item \gls{MIB} consistency throughout the cluster upon network topology changes
	\item Network programming accuracy
\end{enumerate}
%
For the request router specific functional tests, the key points tested where as follows:
\begin{enumerate}
	\item Request router clustering and state replication
	\item Integration with \gls{LVS} through the ipvsadm tool
	\item Integration with the \gls{SDN} controller elastic cluster membership
	\item Reconfiguration of \gls{LVS} through the ipvsadm tool upon controller elastic cluster membership
\end{enumerate}
%
%Preliminary tests of this experimental implementation show that the proposed architecture provides the desired functionalities.
%These tests showed that the MIB is kept consistent throughout cluster members after adding, removing and provoking deliberate failures on controller instances as well as OpenFlow switches and request routers.
%All the relevant network events (such as switch, host, links and adjacencies) are being detected, properly handled and triggering the appropriate message propagation to cluster members.
%
\subsection{Results}
\label{section:functional-tests-results}
%
The following describe the results obtained for the \gls{SDN} controller elastic cluster specific functional tests:
\begin{enumerate}
	\item 
	\item 
	\item 
	\item 
	\item 
	\item 
\end{enumerate}
%
The following describe the results obtained for the request router specific functional tests:
\begin{enumerate}
	\item 
	\item 
	\item 
	\item 
\end{enumerate}
%
\section{Perfomance tests}
\label{section:performance-tests}
For the \gls{SDN} controller elastic cluster specific performance tests, the key points tested where as follows:
\begin{enumerate}
	\item 
\end{enumerate}
%
For the request router specific performance tests, the key points tested where as follows:
\begin{enumerate}
	\item 
\end{enumerate}
%
%
\subsection{Scenarios}
\label{subsection:performance-tests-scenarios}

%
\subsection{Results}
\label{subsection:performance-tests-results}

%
\section{Considerations}
\label{section:considerations}
* VMs running on uncontrolled hardware -> dont know how overloaded it is -> relative results
* Interpret Functional and Performance results